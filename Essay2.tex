\documentclass[conference]{IEEEtran}
\usepackage{setspace}
\usepackage{cite}
\usepackage{amsmath,amssymb,amsfonts}
\usepackage{graphicx}
\usepackage{textcomp}
\usepackage{xcolor}
\def\BibTeX{{\rm B\kern-.05em{\sc i\kern-.025em b}\kern-.08em
    T\kern-.1667em\lower.7ex\hbox{E}\kern-.125emX}}	
\doublespacing
\begin{document}

\title{Testing and Verification in Agile Methodology}

\author{\IEEEauthorblockN{Matthew Bevis}
\IEEEauthorblockA{\textit{Department of Computer Science}\\
\textit{Florida State University}\\
Tallahassee, Florida USA\\
mb19bo@my.fsu.edu}}
\maketitle
\begin{abstract}
The Agile method is one of the most widely implemented software development processes due to its adaptability, and its capability to anticipate the needs of the customer at incremental steps in the development process.  This methodology can save software development teams weeks worth of time and money and aide in the design process along the way.
\end{abstract}
\section{Introduction}
The Agile methodology was developed and adopted by a summit of 17 software developers beginning in 2001.  The goal in mind of these developers was to create a new software development ideology that would surpass the status quo that had been the utilization of the Waterfall method, where testing was done after all parts of a project had been complete.  The issue being addressed was the ineffectiveness of other methods of software engineering including the Waterfall method, in that in many cases after the entire project had been coded to the original specifications of the customer, the customer would see faults in the end result and request reworking of key components often causing the project to take much longer than anticipated.\cite{HoA}  With the advent of the Agile method, much of this was resolved by moving testing up in the process and isolating components of the project for testing at intermittent steps and reevaluating the design based on feedback from testing.  While Agile is the over-arching model that many companies look to for guidance with leading teams of software engineers, Agile is merely the framework and serves as a guide for many different ideologies regarding how teams should meet and coordinate.  SCRUM, Extreme Programming (XP), Kanban, Lean Development, and Crystal are among the most widely used Agile methodologies.  Each separate methodology puts emphasis on different parts of the process, and some, like Lean Development require trimming unnecessary members from the process in order to emphasize efficiency.   Agile methodologies have now become some of the most widely used practices for software development to date.  In this report, the software process, examples of the Agile processes at work within companies, the composition and organization of software teams using an Agile methodology, and the future of testing and verification will be discussed further.
\section{Software Process}
The software development process using the Agile methodology is to test early and often, isolating each component of a program and running various tests to work out any issues as they arise.  This requires testing and verification at regular intervals, rather than waiting to present a completed product to the customer.  This is beneficial to both developers and the customer, as it creates a dialogue to what works, and what does not.  By testing and verifying at each stage, identifying issues as the come, and keeping everyone working with the project in the loop, the result is a finished product that looks and acts just as the customer intends and in a timely manner.  A key component to Agile testing is to gather groups together, tackle issues as a collective unit, rather than as individual specialized teams, and ensure that everyone developing the project is meeting on a regular basis to maintain a good pace and be as transparent as possible about what expectations are.  The Agile methodology is am effective way to get specialized groups of software developers to come together and have a conversation about the state of affairs for the project on a regular basis, build a dialog on how things are going at present, and how any issues that may arise be dealt with and mitigated moving forward.  The Agile method is designed to be a cyclical process that never is truly finished as most software development teams in this modern age have to adapt to needed changes, new design principals, and new technology coming forward.  The software development process rarely ends after first release, as new updates to operating systems come forward, new ideas for other aspects of the project arise, and even just regular user interface changes--just to name a few of the needs for further development that may come to light.  The core of the an Agile system typically involves a discussion phase, where requirements are analyzed and time-frames are in the conversation.  The discussion phase is then followed by the design phase, where requirements are outlined for implementation.  The next phase involves the development of said requirements, and the eventual evolution of an iteration for the next phase of testing.  After testing, the project is moved on to deployment, where the software is subject to feedback.  Based on the feedback, further tweaking the software is considered, and the process restarts from the beginning with another discussion period regarding the project moving forward. 
\begin{center}
\includegraphics[scale=.55]{agilemodel.png}\\
\cite{SCRUM}
\end{center}

\section{Use Cases}
Agile methodologies all centrally revolve around testing as soon as possible, and as often as possible, meaning that testing is the root of the success of each Agile methodology.  The Agile testing method is typically spearheaded by doing acceptance testing to make sure that the features of the software meet the required specifications. Afterwards white box testing normally would occur as soon as the development portion of the project is completed, and specifications are met.  By starting with acceptance testing, then unit testing, early bugs can be detected and dealt with before being presented to shareholders to ensure that a quality product is being created.  After automated unit testing is complete, the process then would typically move on to more exploratory testing, whether that be white box, grey box, black box testing, or fuzzing as well.  It is important to ensure that all test cases are covered by a variety of different standpoints to make certain that no edge cases are ignored, and that software is of the highest quality.  After all testing is complete, it is time again to reevaluate, assess if any other features need to be added, and if necessary start the design phase anew.  If adding new features to the software, another important step in the development phase is rigorous regression testing.  Regression testing is done to ensure that new features do not impede any existing features, and that every part of the software plays nicely with the others, otherwise the code needs to be reworked, or possibly even redesigned.\cite{REG}  Many companies have benefited from switching to the Agile testing method, and have seen an overall increase in productivity, adaptability, and quality of software being developed.  One such company that switched to an Agile methodology in 2018 is \textit{LEGO Group}, which in the year following saw huge reductions in time committed to responding to change.  \textit{LEGO} saw this time reduced from months to weeks in enacting changes within their fundamental duties.  By adopting Agile methodologies, \textit{LEGO} was able to increase work efficiency and productivity by increasing transparency in all facets of their company.\cite{LEGO}  By co-creating, and working together in the development process, employees feel heard and respected, which also increases morale.
\section{Software Team Composition and Organization}
To reiterate, transparency is one of the key factors of Agile methodology that makes it so effective.  This means that not only are engineers working along-side testers, but feedback going back and forth is critical.  By meeting routinely, and hearing how testing goes, and how the development process is going, testing can be done more efficiently through better understanding of the underlying code, and developers can gain a better understanding of how testing is going, and what to possibly anticipate when developing.  Communication is done frequently, with some Agile methodologies meeting daily, as is the case with XP and SCRUM models, with all respective teams.\cite{ASS}  Since testers and developers are working so closely together, the planning of the project can work congruently. 
\section{The Future of Testing and Verification}
One thing that makes Agile methodologies work is the adaptability of the process to be able to be moulded to fit any business model effectively.  There are many different frameworks that already exist that have been adapted from the original Agile model, that were discussed earlier, such as the SCRUM and XP methods.  Automation of testing is something that has become more of a status quo for testing in recent years.  By automating many parts of testing, time can be saved from having to design new tests for similar types of software.  As software gets more complex, new forms of testing will need to be adopted as well to keep up, however, because the Agile method is so versatile, these new strategies can be assimilated into existing structures.  Other software development structures such as the Waterfall method are rapidly losing traction as software complexity increases, and the time necessary to do proper testing increases.  These issues have already made themselves prevalent, and many companies have already seen the results of switching to Agile methods.  Agile works mainly due to the fact that everyone involved is in the loop with the development strategy and more involved in the process.
\section{Conclusion}
The Agile method will most likely continue to be the premiere model to follow for software development due to its adaptability, and effectiveness in keeping teams on track and with the same goal in mind.  Agile has been proven to be effective in many different aspects, as can be shown through many case studies done on companies who have adopted the process, including building block titan, \textit{LEGO}.  Through regular meetings, a structured design, development, testing, and customer conference phases before the final product is released, companies can ensure that quality software built to the exact specifications of the customer.  Although the software design process is constantly changing due to advances in technology, and software complexity, Agile methodologies are intuitive and adaptable.  Agile methodologies are capable of standing up to changes in the software design process, because they are merely a framework for how software development teams should communicate and all contribute to the effective design and structure of software.

\bibliographystyle{ieeetr}
\bibliography{references}

\end{document}
